\documentclass[a4paper,10pt]{book}
\usepackage[utf8x]{inputenc}
\usepackage{amsmath} 
\usepackage{amssymb} 
%\usepackage{makeidx} 
\usepackage{graphicx}
\usepackage{tabularx}


\newcommand{\image}[3]{
\begin{figure}[h!]
 \begin{center}
 #3
 \end{center}
 \caption{#2}
 \label{#1}
\end{figure}
}

\newcommand{\moduledoc}[2]{
  %\section{\bundlefile{lecturesight-objecttracker-impl.jar}

Bundle description.

\configproperties

\property{cv.lecturesight.regiontracker.maxnum}{30}{
$[1 \dots \text{MAX\_INT}]$ Maximum number of regions that should be tracked.}

\property{cv.lecturesight.objecttracker.simple.width.min}{15}{Property description}

\property{cv.lecturesight.objecttracker.simple.width.max}{60}{Property description}

\property{cv.lecturesight.objecttracker.simple.height.min}{12}{Property description}

\property{cv.lecturesight.objecttracker.simple.height.max}{100}{Property description}

\property{cv.lecturesight.objecttracker.simple.ttl}{600000}{Property description}

\property{cv.lecturesight.objecttracker.simple.regionmatch.threshold}{5.0}{Property description}

\property{cv.lecturesight.objecttracker.simple.chthreshold}{0.12}{Property description}

\property{cv.lecturesight.objecttracker.simple.distancethreshold}{15.0}{Property description}

\property{cv.lecturesight.objecttracker.simple.channel.number}{4}{Property description}}
  \section{#1}                                          % not so nice work-around for problem with line above
  \bundlefile{lecturesight-objecttracker-impl.jar}

Bundle description.

\configproperties

\property{cv.lecturesight.regiontracker.maxnum}{30}{
$[1 \dots \text{MAX\_INT}]$ Maximum number of regions that should be tracked.}

\property{cv.lecturesight.objecttracker.simple.width.min}{15}{Property description}

\property{cv.lecturesight.objecttracker.simple.width.max}{60}{Property description}

\property{cv.lecturesight.objecttracker.simple.height.min}{12}{Property description}

\property{cv.lecturesight.objecttracker.simple.height.max}{100}{Property description}

\property{cv.lecturesight.objecttracker.simple.ttl}{600000}{Property description}

\property{cv.lecturesight.objecttracker.simple.regionmatch.threshold}{5.0}{Property description}

\property{cv.lecturesight.objecttracker.simple.chthreshold}{0.12}{Property description}

\property{cv.lecturesight.objecttracker.simple.distancethreshold}{15.0}{Property description}

\property{cv.lecturesight.objecttracker.simple.channel.number}{4}{Property description}
}

\newcommand{\configproperties}{
  \subsection{Configuration Properties}
}

\newcommand{\property}[3]{
  \setlength{\tabcolsep}{0pt}
  \noindent\begin{tabularx}{\textwidth}{l}
    \textbf{#1}\\\hline\hline
    \noindent {\small Default: \texttt{#2}}\\\hline
  \end{tabularx}
  \setlength{\tabcolsep}{5pt}
  \vspace{1pt}\\
  #3
  \vspace{15pt}
}

\newcommand{\bundlefile}[1]{
  \setlength{\tabcolsep}{0pt}
  \begin{tabularx}{\textwidth}{l}
    Bundle File: \textbf{#1}\\\hline
  \end{tabularx}
  \setlength{\tabcolsep}{5pt}
}

\newcommand{\consolecommand}[3]{
  \setlength{\tabcolsep}{0pt}
  \noindent\begin{tabularx}{\textwidth}{l}
    \textbf{#1}\\\hline\hline
    \noindent {\small\texttt{#2}}\\\hline
  \end{tabularx}
  \setlength{\tabcolsep}{5pt}
  \vspace{1pt}\\
  #3
  \vspace{15pt}
}

\begin{document}

\author{B. Wulff, A. Fecke}
\title{LectureSight Manual}

\frontmatter
\tableofcontents

\mainmatter

%% ============== Operation Manual ============== %%
\chapter{Operation Manual}

This chapter is intended to be the actual manual holding all the knownledge about operating a LectureSight instance in a lecture hall or seminar room.

\section{Getting Started}

This section should serve as a step-by-step guid to setting up and running LectureSight in a productive envoironment. First discussed are the prerequisits of hard- and software. The reader may skip section \ref{subsec-prerequisits} completely if the host system has already been set up.

\subsection{Prerequisits}
\label{subsec-prerequisits}

LectureSight is a software build up of modules for the OSGI software modularization framework \cite{alliance2007osgi} for the Java programming language. Java itself is platform-independent but LectureSight relies on certain features of the Linux operating systems, so the software must be run on Linux. Also LectureSight uses the GPU for the video ananlysis and thus a fairly modern graphics card will be needed to run the system. Also the choice of the overview camera is can impact tracking performance.

The next section gives an in-depth discussion about hardware  components that can be used with LectureSight. The following section describes the software requirements for running LectureSight.

\subsubsection{Hardware}

As mentioned before, as the host system for LectureSight a fairly modern personal computer that is capable of running Linux is sufficient.

~\\\noindent\textbf{CPU}\\
\noindent LectureSight was designed with the goal of being capable of running on a single 2 GHz core. The system should not take up all CPU resources in a modern system so that video recording software can run alongside on the same computer. Thus LectureSight should run on any modern system. The suggestion is to use a system with a CPU of at least the performance of an Intel Core 2 Duo with 2,2 Ghz.

~\\\noindent\textbf{RAM}\\
\noindent Since video ananlysis is nearly entirely done on the graphics card, there are no special memory requirements for LectureSight when run stand-alone. The usual defualt memory configuration for the Java VM suffice.  

~\\\noindent\textbf{Mass Storage}\\
\noindent For the whole system roughly 50 MB of mass storage must be reserved. Despite of log files that may need to be rotated, there are no sites where data is incrementally saved.

~\\\noindent\textbf{GPU}\\
The video processing portions of the system haven nearly entriely been implemented for the GPU using the OpenCL standard for corss-platform parallel computing. Most modern graphics cards are compatible with OpenCL. Graphics cards with Nvidia GPU that are label as ``CUDA compatible'' are OpenCL compatible. ATI graphics chips that are labeled ``Stream SDK compatible'' are also compatible with OpenCL.

For use in a real-time scenarion it is suggested to use a GPU with a least six OpenCL \textit{compute units} and at least 512 MB of graphics memory. Such a configuration can be found, for example, in grpahics card equipped with the Nvidia GT 220 chip set. For Nvidia GPUs the number of \textit{CUDA units} divided by eight yields the number of OpenCL compute units. The GT 220 a number of 48 CUDA units is stated, thus the GPU provides 6 compute units in OpenCL.

~\\\noindent\textbf{Overview Camera}\\
As overview camera USB webcams as well as analog SD video cameras connected to a fast frame grabber device can be used. 

Though the resolution for the overview camera should be chosen not too high in order not to jeopardize real-time performance (usually VGA), the image qualtiy of the model chose as overview camera directly impacts tracking accuracy and reliability. Cheap USB webcams, for example, sometimes show a habit of aggressively adjusting color channels. Such behavior can compromise correct function of the tracker. With higher quality (720p) web-cams color and contrast stable images can be achieved which is optimal for the video analysis. 

Also most stationary analog video cameras produce a stable image that suffices the needs of the tracking algorithms. In order to use an analog video camera as the overview camera, a frame grabber has to convert the analog signal to digital frames. It is suggested to use interal PCI(e) frame grabber hardware with direct memory access (DMA). Such frame grabbers write raw frame directly into the host systems memory from where they can directly be copied to GPU memory, thus avoding unnecessary memory copy or encoding/decoding operations that can produce delays.

\subsubsection{Software}

The OSGI-bundles that make up LectureSight come with all their dependencies included. Thus there is no need for third-party software to be installed on the host system except the OpenCL implementation. The OpenCL implementation is usually part of the driver package that comes with the graphics card. For Nvidia products the Linux driver can be downloaded from the website
\\~\\
\texttt{http://www.nvidia.com/object/unix.html}
\\~\\
For ATI products the Linux driver can be downloaded from
\\~\\
\texttt{http://support.amd.com/de/gpudownload/linux/Pages/radeon\_linux.aspx}
\\~\\
The driver installations include the OpenCL library implementation suitable for the particular graphics card and should install an \texttt{.icd} file under\\ \texttt{/etc/OpenCL/vendors} that points to the library file.

\subsection{Installation}

The software can be obtained as pre-compiled binary packages or as source code checked out from the Subversion repository via the following command line
\\~\\
\texttt{svn co http://opencast.jira.com/svn/LECTURESIGHT/trunk}
\\~\\
The pre-compiled binary distribution for production systems can be downloaded via the following command line
\\~\\
\texttt{wget https://opencast.jira.com/...}
\\~\\
Extract the software to \texttt{/opt/lecturesight}
\\~\\
\texttt{tar xfvz lecturesight-0.3.1-production.tar.gz}\\
\texttt{mv lecturesight /opt}
\\~\\
Change into the lecturesight directory and start the software
\\~\\
\texttt{cd /opt/lecturesight}\\
\texttt{./bin/start\_lecturesight.sh}
\\~\\


\subsection{Configuration}

\section{Advanced Configuration}

\subsection{Video Analysis}

\subsection{Tracking}

\subsection{Adjustment to Room Conditions}

\subsection{Camera Operator Scripting}

%% ============== Development Manual ============== %%
\chapter{Development Manual}

%% ============== Module Documentations ============== %% 
\chapter{Module Documentations}

\section{API Modules}

This section is provides the documentation for the API bundles found in LectureSight. APIs have been defined inside bundles of their own so that implementations can be added or exchanged later on.

This section may be of special interested for developers. Operators, however, will in most cases not find valuable information here. 

\section{Implementation Modules}

In this section detailed manuals about all bundles that hold implementations of services in LectureSight can be found. The subsections explain the functionalities of those modules  and give insight into the inner workings of the services. Further more all configuration parameters and console commands for every service are lsited and explained. Operators may use this section as a reference for deployment and configuration.

\moduledoc{Frame Source Manager}{framesource-impl}
\moduledoc{Video4Linux FrameSource}{framesource-v4l}
\moduledoc{Video File FrameSource}{framesource-videofile}
\moduledoc{Video Analysis Services}{videoanalysis-impl}
\moduledoc{Foreground Region Tracker Service}{regiontracker-impl}
\moduledoc{Object Tracker Service}{objecttracker-impl}
\moduledoc{Decorator Color Histrogram}{decorator-color}
\moduledoc{Decorator Head}{decorator-head}
\moduledoc{VISC Protocol PTZ Driver}{ptzcontrol-visca}
\moduledoc{iCal Scheduler}{scheduler}

\backmatter
%\include{glossary}
%\include{notat}
\bibliographystyle{amsalpha} 
\bibliography{references} 
%\printindex

\end{document}