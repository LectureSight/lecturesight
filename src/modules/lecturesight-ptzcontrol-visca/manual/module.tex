\bundlefile{lecturesight-ptzcontrol-visca.jar}

The bundle provides a driver for cameras speaking the VISCA protocol defined by Sony. On activation the service tries to initialize all VISCA cameras on a configured serial devices. Upon discovery of a VISCA camera the driver determines camera vendor and model and tries to load a fitting device profile. If no fitting profile is existing the driver loads a default profile that has the same configuration as the profile for the Sony EVI-D30. Most VISCA cameras can be configured to run in D30 compatibility mode. 

\configproperties

\property{cv.lecturesight.ptz.visca.ports}{\texttt{none}}{A comma-separated list of serial devices that are connected to one or more VISCA cameras (ie. \texttt{/dev/ttyS0}). If this property value is empty the service will not activate.}

\subsubsection{Camera Profiles}

The following is an example of a camera profile definition. It is the default camera profile. Thats why the values for \texttt{camera.vendor.id} and \texttt{camera.model.id} are set to \texttt{DEFAULT}. In actual camera profiles the values are numeric (byte) values.

\begin{verbatim}
camera.vendor.id=DEFAULT
camera.vendor.name=ACME Inc.
camera.model.id=DEFAULT
camera.model.name=RoboCam
camera.pan.min=-880
camera.pan.max=880
camera.pan.maxspeed=18
camera.tilt.min=-300
camera.tilt.max=300
camera.tilt.maxspeed=14
camera.zoom.min=0
camera.zoom.max=16
camera.zoom.maxspeed=7
camera.home.pan=0
camera.home.tilt=0
\end{verbatim}
