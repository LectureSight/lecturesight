\label{module-framesource-v4l}
\bundlefile{lecturesight-framesource-v4l.jar}

The bundle provides a FrameSource implementation for accquiring frames from Video4Linux and Video4Linux 2 devices. Arguments for creation of a new FrameSource from this implementation can be provided in the FrameSource MRL. If an argument is not present in the MRL, the default value is take from the configuration properties.

\subsubsection{Usage}

The \texttt{type} for this FrameSource implementation is \textbf{v4l}. The \texttt{path} is the path to a linux video device (\texttt{/dev/videoX}). Availabel arguments are \texttt{width, height, standard, channel, quality}, their meanings are the same as those of the properties in the next section \ref{cfgprop-v4lframesource}.
\\~\\
Example usage: \hspace{30pt} \texttt{v4l:///dev/video0[width=320;height=240]}
\\~\\
The example MRL tells the system to create a FrameSource using a Video4Linux device \texttt{/dev/video0} as input with QVGA resolution.

Example usage: \hspace{30pt} \texttt{v4l2:///dev/video0[width=320;height=240]}
\\~\\
The example MRL tells the system to create a FrameSource using a Video4Linux2 device \texttt{/dev/video0} as input with QVGA resolution.

\configproperties
\label{cfgprop-v4lframesource}

\property{cv.lecturesight.framesource.v4l.resolution.width}{320}{Default width for input frames.}

\property{cv.lecturesight.framesource.v4l.resolution.height}{240}{Default height for input frames.}

\property{cv.lecturesight.framesource.v4l.standard}{0}{Default video standard. Usually not used with USB webcams but rather with capture cards. Which value indicates a certain standard (eg. PAL-X/NTSC) depends on the driver of the video device.}

\property{cv.lecturesight.framesource.v4l.channel}{0}{Default video input. Usually not used with USB webcams but rather with capture cards. This can be usefull with capture cards, since they are by default set to tuner input and need to be set to composite (usually 1).}

\property{cv.lecturesight.framesource.v4l.quality}{0}{Default encoding quality. Only used for devices that provide encoded video streams (such as MPEG2 or MJPEG). Value range depends on device driver.
\\~\\
\noindent\textbf{Note: For real time operation only devices that provide raw video streams should be used since encoding and decoding of frames can lead to several hundred milliseconds of delay.} 
}
